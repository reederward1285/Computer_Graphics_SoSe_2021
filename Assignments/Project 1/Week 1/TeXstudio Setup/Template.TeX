\documentclass[a4paper,pdftex,12pt]{article}
\usepackage[T1]{fontenc} % utf8 <- produce real utf8 characters
\usepackage[utf8]{inputenc} % utf8 <- accept utf8 input characters
%\usepackage[german]{babel}

\usepackage[hscale=0.75,vscale=0.75,vmarginratio={85:100},heightrounded]{geometry} % less margin at bottom

\usepackage{graphicx}
\usepackage{latexsym}
\usepackage{amsmath, amssymb}
\usepackage{color, eurosym} % 6.9.07
\usepackage{float}
\usepackage{hyperref}
\usepackage{xspace} % set a space if not fullstop / end of sentence
\usepackage{times}

% Seitenformatierungsbefehle
\setlength{\textheight}{220mm}
\setlength{\textwidth}{150mm}
\setlength{\topmargin}{1mm}
\setlength{\headheight}{0mm}
\setlength{\headsep}{0mm}
\setlength{\oddsidemargin}{5mm}
\setlength{\parindent}{32mm}
\setlength{\parskip}{0mm}
\linespread{1.1}

\sloppy  % verhindert, dass Wörter über den Rand herausragen


% Makros
\newcommand{\inv}[1]    {\frac{1}{#1}}
\newcommand{\half}      {\frac{1}{2}}
\newcommand{\R}{{\mathbb R}}
\newcommand{\sect}[1] {\overline{#1}}
\newcommand{\eqn}[2] {\begin{equation} \label{#1} #2 \end{equation}}
\newcommand{\eqnn}[1] {\begin{equation*} #1 \end{equation*}}

\title{% \vfill
    %\vspace{-2.0cm}
    Place The Title Here}
\author{
    Student1, Student2, Student3 \\ Media Computer Science\\ Hochschule Bremen City University of Applied Sciences}
\date{ \today}
\parindent 0pt
\parskip 1ex


\begin{document}
%\twocolumn

\maketitle
\begin{abstract}
	A very short summary of the contents and results. No introduction to the topic.
\end{abstract}
%\vspace{-1.0 cm}

%\parskip 1ex

%\newpage



\section{Bézier Curves}

YadiYadiYadi.\\

Formulae can be written within the text $ x = \sum_{i=0}^9 i^2$.
or even better as numbered equations:
\begin{equation} \label{eqnX}  
    x = \sum_{i=0}^9 i^2
\end{equation}

Equation (\ref{eqnX}) is only "`bogus"', of course. Here we have a matrix:
\begin{equation} \label{eqnSys}
    \left[ \begin{array}{c} \dot y \\ \dot v \end{array}\right]
    = \left[ \begin{array}{cc} 0 &1 \\ -13 &6 \end{array}\right]\
		\left[ \begin{array}{c} y \\ v \end{array}\right]
\end{equation}

Matrices can be used to define the Bézier Points. Alternatively, use the 
verbatim environment described in the next section.

  
\begin{figure}[htb]
\centering
  Image Template:\\ The following line\\ within the code $\% $-sign\\ places a bitmap as floating object:
  % \includegraphics[width=8.0cm]{dateiname.png}\\
\it \caption{\label{abbMech}
	A simple mechanical system containing two springs or whatsoever. 
}
\end{figure}
	
	
\subsection{Matlab-Code}
(Use Section 1.1 only when 1.2 exists, also!)

Code can be listed as follows:
The Matlab implementation plotting the eigen solution of our simple differential equation reads as follows:
\begin{verbatim}
A = [0 1; -13 6];
[E, L] = eig(A);
lambda = L(1,1);               % 1st eigen value
r = E(:,1);                    % 1st eigen vector
t = 0:.01:2;                   % very short time interval
y = r * exp( lambda*t);        % 1st eigen function
plot( t, real(y), t, imag(y))  % real und imaginary paarts for y und v
\end{verbatim}

A short description is always good. Note that there exist alternative environments for listings, as well.

Since figures are implemented as floating objects, each of them should be referred to at least once
within your text, for example, see figure  \ref{abbMech}.
Literature is not referrenced with $\backslash$ref, but rather using
$\backslash$cite like the book by Gerald Farin \cite{Far02}. Online sources should not be mixed with referred punlications.
These can be contained in an extra listing or simply placed as
footnotes \footnote{\tt https://de.wikipedia.org/wiki/Bezierkurve}.

\subsection{Numerical Examples}

\section{Conclusions and Future Work}
It should be recognized that the Euler method is very imprecise.
Thus, an integration method with greater approximation order, like that by Runge-Kutta is often preferred.

\begin{thebibliography}{20} % the 20 defines the maximal width used by the labels
\bibitem{Far02} 
Gerald Farin, {\it Curves and Surfaces for CAGD. A practical guide.}
5th edition, Academic Press, San Diego 2002, ISBN 1-55860-737-4
\end{thebibliography}
\end{document}
